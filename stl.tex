\section{标准模板库}

相比于 C 语言,C++ 有更为丰富的标准库设施。作为标准库的一部分,标准模板库(Standard Template Library,STL)提供丰富的容器和算法接口,并且易于使用。由于 STL 是一套非常庞大的设施,本节的目标仅仅是带领读者认初步了解 STL 设施的使用并认识到它们的价值,进一步的学习需要阅读更详细的参考资料。。如果希望深入学习,读者可以阅读文献\cite{Lippman2013C,StlRef2002,HouJie2002STL},建议读者在已经熟悉 STL 的情况下再去阅读文献\cite{HouJie2002STL}。

STL 有两种常用的设施:\textbf{容器}和\textbf{算法}。后面我们还会看到,通过\textbf{迭代器}可以将两者结合起来。

\subsection{容器}

\textbf{容器} 是存储对象集合的对象,常见的 \texttt{vector}、\texttt{list} 等都是容器。C++ 标准对容器的实现有一些规范,例如要求 \texttt{vector} 支持变长序列的常量时间随机访问,且对尾部的插入时间开销也要是常量的。从另一个角度来说,我们知道容器的这些特性即可,而不用关心容器的实现方式(各厂家的实现也不尽相同)。

继续以 \texttt{vector} 作为例子,我们演示一些基本的操作:
\begin{lstlisting}[language=c++]
std::vector<int> v; // 定义一个存储 int 数据的 vector 对象
v.push_back(1); // 在 v 的尾部插入一个数值 1
v.insert(v.begin() + 1, 2); // 在 v 的第 1 个元素处插入 2
printf("%zu", v.size()); // 打印 v 的元素数量为 2
printf("%d, %d", v[0], v[1]); // 1, 2
v[0] = 3; // 将 v 的第 0 个元素修改为 3
\end{lstlisting}
上面演示了 \texttt{vector} 的定义、插入和随机访问等操作。接下来我们看 \texttt{list} 容器的相似操作:
\begin{lstlisting}[language=c++]
std::list<int> l; // 定义一个存储 int 数据的 list 对象
l.push_back(1); // 在 l 的尾部插入一个数值 1
printf("%zu", l.size()); // 打印 l 的元素数量为 1
// 错误:list 不支持随机访问,不能通过”l 的第 n 个元素“这种方式来指定访问位置,因此不支持下标访问
l.insert(l.begin() + 1, 2);
l[0] = 3;
\end{lstlisting}

\subsection{迭代器}

\subsection{算法}
