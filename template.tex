\section{模板}

回想一下,我们在 C 中实现一个判断两个对象大小的函数,一般要针对不同的类型编写不同的函数:
\begin{lstlisting}[language=c++]
// 求较大整数的函数
int max_int(int a, int b) {
    return a > b ? a : b;
}
// 求较大浮点数的函数
double max_double(double a, double b) {
    return a > b ? a : b;
}
// 使用宏实现的 max 算法
#define max_type(a, b)  ((a) > (b) ? (a) : (b))
\end{lstlisting}
前面两个 \texttt{max\_xxx} 函数的缺点非常明显:我们必须针对每种类型重新编写一份几乎一样的代码。而宏版本的缺点也很明显:容易对参数重复计算(宏本质是字符串替换),同时功能复杂的宏很难维护。在 C++ 中我们可以使用 \textbf{模板} 来实现这种核心相同,但是要用于不同类型或者场合的算法。

\subsection{函数模板}

作为第一个例子,我们尝试通过 \textbf{函数模板} 实现上面的 \texttt{max} 函数:
\begin{lstlisting}[language=c++]
template <typename T>
T max(const T &a,  const T &b) {
    return a > b ? a : b;
}
\end{lstlisting}
模板的定义从 \textbf{\texttt{template}} 关键字开始,后面的尖括号对中是\textbf{模板参数列表},多个模板参数之间使用逗号分隔。例子中的模板参数 \texttt{typename T} 定义了一个类型 \texttt{T},它代表一个类型,并在模板被使用时被转换为一个实际的类型。我们通过这个模板的使用方法来说明模板参数:
\begin{lstlisting}[language=c++,numbers=none]
double v = max<double>(1.0, 3.0);
\end{lstlisting}
为了计算两个 \texttt{double} 值的最大者,需要指定 \texttt{double} 作为\textbf{模板实参}。编译器会为 \texttt{max<double>} 生成这样的等效函数:
\begin{lstlisting}[language=c++]
// 使用 double 代替模板参数 T
double max<double>(const double &a, const double &b) {
    return a > b ? a : b;
}
\end{lstlisting}
函数模板被调用的同时会被\textbf{实例化}。通常,编译器会根据函数实参来推断模板实参,然后使用模板实参替换对应的模板参数。对于前面的 \texttt{max} 函数,下面的使用方式也会生成 \texttt{max<double>} 实例:
\begin{lstlisting}[language=c++,numbers=none]
double v = max(1.0, 3.0); // 编译器会自动推断函数模板实参
\end{lstlisting}

\subsection{模板参数}

前面的例子中,函数模板 \texttt{max} 有一个被 \texttt{typename} 修饰的参数 \texttt{T},这表示 \texttt{T} 是一个\textbf{类型参数}。使用 \texttt{typename} 或者 \texttt{class} 修饰的模板参数都是类型参数,在模板参数中,这两个关键字含义相同并可以互换使用。类型参数的模板实参必须是一个类型。

我们还可以使用 \textbf{非类型参数},使用类型名修饰模板参数来指定非类型参数:
\begin{lstlisting}[language=c++]
// 如果 x 小于 lower,返回 lower;如果大于 upper,返回 upper;否则返回 x
template <int lower, int upper>
int bound(int x) {
    return min(max(lower, x), upper);
}
\end{lstlisting}
我们可以这样使用函数模板 \texttt{bound}:
\begin{lstlisting}[language=c++,numbers=none]
int v = max<0, 10>(3);
\end{lstlisting}
非类型模板参数的实参必须是常量表达式,因此其实参是一个常量值。需要使用常量参数的场合可以通过非类型模板参数来实现,例如上面的 \texttt{bound}。

\subsection{类模板}


我们还能举出一些例子,比如链表。链表的算法基本都是相同的,但是我们要在链表中存储不同的数据类型。对此一般有两种实现思路:
\begin{itemize}
  \item 数据类型必须派生自链表节点类,并通过结点类维护链表;
  \item 为每种数据类型实现链表算法。
\end{itemize}
第一种方法的思路在 C 中使用十分广泛,这种用法的缺点是需要侵入式修改并且须要使用指针来访问数据,使用上并不友好;第二种实现则不可能作为库中的代码。作为对比,我们看下 C++ 标准库中的链表使用方法:
\begin{lstlisting}[language=c++]
std::list<int> l;       // 定义存储 int 数据的链表
l.push_back(1);         // 在尾部插入 1
l.push_front(2);        // 在头部插入 2
l.insert(--l.end(), 3); // 在倒数第 2 个节点后插入 3
l.erase(++l.begin());   // 删除第二个元素
l.pop_back();           // 删除尾部元素
\end{lstlisting}
可以看出,标准库将链表的实现封装的非常好,这些代码看起来就是在一种``容器''中操作我们存储的 \texttt{int} 数据,而不用关心这些数据是怎样被存储的,因此容易使用。

在 C++ 中,通过模板可以将一份代码用于不同的类型或某些场景
